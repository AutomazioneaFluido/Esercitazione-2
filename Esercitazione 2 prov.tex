\documentclass[a4paper]{article}
\usepackage[T1]{fontenc}
\usepackage[utf8x]{inputenc}
\usepackage[italian,english]{babel}
\usepackage{amssymb,latexsym,amsfonts,amsmath}
\usepackage{lipsum}
\usepackage{url}
\usepackage{graphicx}
\usepackage {pdfpages}

\begin{document}


\title{Esercitazione 2}
\date{April 5 , 2017}
\maketitle


\author{Alessio Susco \hspace*{2cm} Nicola Bomba \hspace*{2cm} Fabrizio Ursini  \\  \hspace*{1,85cm} Alessandra Di Martino \hspace*{1,25cm} Diego Guzman}

\includepdf[pages={1,2,3,4},pagecommand={\thispagestyle{plain}}]{eserc2.pdf} 

\tableofcontents

\clearpage

\section{Introduzione Generale}

Questa esercitazione si sviluppa in sei prove. Nelle seguenti prove bisogna effettuare delle verifiche per constatare il corretto funzionamento di diverse valvole:
\begin{enumerate}
\item Verificare il funzionamento di una valvola monostabile;
\item Verificare il funzionamento di una valvola bistabile con segnali positivi di pressione;
\item Verificare il funzionamento di una valvola bistabile in mancanza di segnali (comando negativo);
\item Verificare il funzionamento di una valvola monostabile con comando autoalimentato;
\item Realizzare e verificare il funzionamento del limitatore di impulso;
\item Misurare la forza di azionamento di una valvola di fine corsa.
\end{enumerate}



\section{Strumenti Utilizzati}

\subsection{Esercizio 1}

\subsubsection{Prova 1.1}
\begin{itemize}
\item Valvola monostabile 3/2 a comando a pulsante;
\item Valvola monostabile a comando pneumatico 3/2;
\item Lampadina pneumatica;
\item Alimentazione;
\item Tubi in poliuretano.
\end{itemize}

\subsubsection{Prova 1.2}
\begin{itemize}
\item Valvola monostabile a comando pneumatico 3/2;
\item Lampadina pneumatica;
\item Limitatore di pressione x2;
\item Manometro x2;
\item Tubi in poliuretano;
\item Alimentazione.
\end{itemize}


\subsection{Esercizio 2}
\begin{itemize}
\item Valvola bistabile a comando pneumatico 5/2;
\item Limitatore di pressione x2;
\item Manometro x2;
\item Valvola monostabile 3/2 a comando a pulsante;
\item Tubi in poliuretano;
\item Alimentazione.
\end{itemize}

\subsection{Esercizio 3}
\begin{itemize}
\item Valvola bistabile a comando pneumatico 5/2;
\item Valvola monostabile 3/2 a comando a pulsante;
\item Lampadina pneumatica x2;
\item Strozzatore unidirezionale x2;
\item Tubi in poliuretano;
\item Alimentazione.
\end{itemize}

\subsection{Esercizio 4}
\begin{itemize}
\item Valvola monostabile a comando pneumatico 3/2;
\item Valvola monostabile 3/2 a comando a pulsante x2;
\item Manometro;
\item Strozzatore unidirezionale;
\item Regolatore di flusso unidirezionale;
\item Lampadina pneumatica;
\item Tubi in poliuretano;
\item Alimentazione.
\end{itemize}

\subsection{Esercizio 5}
\begin{itemize}
\item Valvola monostabile 3/2 a comando pneumatico;
\item Valvola monostabile 3/2 a pulsante;
\item Lampadina pneumatica x2;
\item Limitatore di impulso;
\item Manometro x2;
\item Tubi in poliuretano;
\item Alimentazione.
\end{itemize}

\subsection{Esercizio 6}
\begin{itemize}
\item Valvola monostabile ad azionamento meccanico bidirezionale;
\item Leva di alluminio;
\item Dinamometro;
\item Limitatore di pressione;
\item Manometro;
\item Tubi in poliuretano;
\item Alimentazione.
\end{itemize}



\section{Osservazione Preliminare}
\subsection{Esercizio 1}
\subsubsection{Prova 1.1}
Per verificare il funzionamento di una valvola monostabile a comando pneumatico, costruiamo un semplice circuito utilizzando una valvola con comando a pulsante e una lampadina pneumatica. Se la valvola monostabile funziona correttamente, la lampadina si accenderà al commutarsi della valvola a comando pneumatico e rimarrà accesa fintanto che teniamo premuto il pulsante.

\subsubsection{Prova 1.2}
Dobbiamo ora realizzare un circuito nel quale azioniamo la valvola monostabile attraverso la regolazione di due diverse pressioni. In questo l’ingresso del segnale di potenza della valvola a comando pneumatico è collegata all'alimentazione, a un regolatore di pressione e a un manometro, così come l'ingresso del segnale di comando $_{12}$. Fissiamo la pressione di alimentazione all'uscita $P_1$ su diversi valori (1,2,3,4,5 bar) e misuriamo per quali valori di $P_{12}$, pressione all'ingresso $_{12}$, la valvola si commuta, e quindi la lampadina si accende, riportando i dati raccolti in una tabella e tracciando un grafico dell'andamento di $P_{12}$ in funzione di $P_1$.


\subsection{Esercizio 2}
Il modo di procedere è analogo al precedente, ma in questo caso dobbiamo verificare il funzionamento di una valvola bistabile a comando positivo. Costruiamo un circuito nel quale colleghiamo l'alimentazione con un manometro e un limitatore di pressione all’ingresso del segnale di potenza della valvola bistabile, l'alimentazione con un altro limitatore di pressione e un altro manometro all’ingresso $_{12}$ del segnale di comando della valvola bistabile e una valvola monostabile a pulsante all’ingresso $_{14}$ del segnale di comando della valvola bistabile. La valvola a pulsante collegata all’ingresso del segnale di comando $_{14}$ serve a dare l’impulso per commutare la valvola bistabile alla posizione iniziale.

\subsection{Esercizio 3}
Ora prendiamo in considerazione il caso in cui una valvola bistabile viene azionata in assenza di segnali di comando in 4 casi:
\begin{enumerate}
\item Entrambe le resistenze chiuse con le valvole 3/2 a riposo o commutate;
\item Entrambe le resistenze a diverse laminazioni e le valvole 3/2 commutate alternativamente;

\item Azionamento simultaneo delle valvole 3/2 di comando con le resistenze a diverse laminazioni;

\item Interruzione e ripristino dell’alimentazione con le resistenze a diverse laminazioni.
Infine per completare questa parte valutiamo il comportamento del circuito sostituendo le resistenze con valvole regolatrici di flusso unidirezionali.
\end{enumerate}

\subsection{Esercizio 4}
In questa parte dobbiamo realizzare un circuito pneumatico nel quale una valvola monostabile viene azionata tramite due valvole monostabili a pulsante e utilizzata come valvola bistabile. Inoltre studiamo il caso in cui azioniamo contemporaneamente le due valvole a pulsante e il caso in cui interrompiamo e ripristiniamo l’alimentazione.
Inoltre, come nella parte precedente, valutiamo il comportamento del circuito sostituendo la resistenza con un regolatore di flusso unidirezionale.

\subsection{Esercizio 5}
Ora dobbiamo verificare il funzionamento del limitatore di impulso diagrammando i segnali $L_1$ e $L_2$ e la pressione $P_{12}$ al variare della strozzatura e della pressione di alimentazione $P_a$. Confrontiamo i valori di $P_{12}$ con quelli della prova precedente.

\subsection{Esercizio 6}
Con l'ausilio di un dinamometro determiniamo la forza di azionamento di due \\
valvole ad azionamento meccanico.\\
 Posizioniamo la valvola sotto una leva incernierata, misurandone la distanza dalla cerniera, e un dinamometro all'estremità della leva, e manualmente applichiamo una forza diretta verso il basso, la quale verrà quantificata dal dinamometro e raccogliamo il valore della forza corrispondente all'azionamento della valvola. La forza misurata dal dinamometro è la forza peso $Q$ espressa in kg peso. Fissiamo una pressione di alimentazione $P_a$ e ricaviamo la forza $F_c$ di commutazione della valvola, riscontrabile osservando l'attivazione di una lampadina pneumatica collegata all'uscita della valvola. Ripetiamo la prova alcune volte e per diversi valori della pressione di alimentazione: 1,2,3,4,5 bar.
Infine per studiare l'andamento della $F_c$, grafichiamo la $F_c$ in funzione di $P_a$.

\section{Schema Circuito}

\subsection{Schema 1.1}
\begin{center}
\includegraphics[scale=1]{S1-1.png}
\end{center}

\subsection{Schema 1.2}
\begin{center}
\includegraphics[scale=0.6]{S1-2.png}
\end{center}

\subsection{Schema 2}
\begin{center}
\includegraphics[scale=0.6]{S2.png}
\end{center}

\subsection{Schema 3}
\begin{center}
\includegraphics[scale=0.6]{S3.png}
\end{center}

\subsection{Schema 4}
\begin{center}
\includegraphics[scale=0.6]{S4.png}
\end{center}

\subsection{Schema 5}
\begin{center}
\includegraphics[scale=0.6]{S5.png}
\end{center}

\subsection{Schema 6}
\begin{center}
\includegraphics[scale=0.6]{S6.png}
\end{center}


\section{Calcoli}
\dots


\section{Descrizione Approfondita dell'Esercitazione}
\subsection{Descrizione Esercizio 1}
...
\begin{itemize}
\item ...
\end{itemize}
...

\subsection{Descrizione Esercizio 2}
\dots

\subsection{Descrizione Esercizio 3}
\dots

\subsection{Descrizione Esercizio 4}
\dots

\subsection{Descrizione Esercizio 5}
\dots

\subsection{Descrizione Esercizio 6}
\dots

\section{Conclusioni}
\subsection{Conclusioni Esercizio 1}
\dots
\subsection{Conclusioni Esercizio 2}
\dots
\subsection{Conclusioni Esercizio 3}
\dots
\subsection{Conclusioni Esercizio 4}
\dots
\subsection{Conclusioni Esercizio 5}
\dots
\subsection{Conclusioni Esercizio 6}
\dots


\end{document}
